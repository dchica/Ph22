%% AMS-LaTeX Created with the Wolfram Language for Students - Personal Use Only : www.wolfram.com

\documentclass{article}
\usepackage{amsmath, amssymb, graphics, setspace}

\newcommand{\mathsym}[1]{{}}
\newcommand{\unicode}[1]{{}}

\newcounter{mathematicapage}
\begin{document}

We can generalize the secant method as\\
\(x_{n+1}=x_n-f\left(x_n\right) \frac{x_n-x_{n-1}}{f\left(x_n\right)-f\left(x_{n-1}\right)}\)\\
Let{'}s suppose we are looking for the solution to \(f(x)=0\), and we find \(f(\xi )=0\), let{'}s define\\
\(x_n=\xi +\epsilon _n\)\\
since this process is iterative, we know that as \(n\to \infty\), \(x\to \xi\), which implies \(\epsilon _{\infty }\to  0\), or the error is very
small. Since we are looking for the order of convergence, this setup leads to the equation\\
\(x_{n+1}-\xi =C\left(x_n-\xi \right){}^r\)\\
which we can express in terms of the error as\\
\(\epsilon _{n+1}=C \epsilon _n^r\)\\
and our goal here is to solve for r.\\
\\
Let{'}s plug in this new definition of \(x_n\) into the secant method\\
\(\left(\xi +\epsilon _{n+1}\right)=\left(\xi +\epsilon _n\right)-f\left(\xi +\epsilon _n\right) \frac{\xi +\epsilon _n-\left(\xi +\epsilon _{n-1}\right)}{f\left(\xi
+\epsilon _n\right)-f\left(\xi +\epsilon _{n-1}\right)}\\
\\
\epsilon _{n+1}=\epsilon _n-f\left(\xi +\epsilon _n\right) \frac{\epsilon _n-\epsilon _{n-1}}{f\left(\xi +\epsilon _n\right)-f\left(\xi +\epsilon
_{n-1}\right)}\)\\
Let{'}s expand the functions using a Taylor Expansion of order 2\\
\(f\left(\xi +\epsilon _n\right)=f(\xi )+f'(\xi )\epsilon _n+\frac{1}{2}f\text{''}(\xi )\epsilon _n^2\)\\
We know \(f(\xi )=0\) from earlier, so\\
\(f\left(\xi +\epsilon _n\right)=f'(\xi )\epsilon _n+\frac{1}{2}f\text{''}(\xi )\epsilon _n^2\)\\
using a similar idea\\
\(f\left(\xi +\epsilon _n\right)-f\left(\xi +\epsilon _{n-1}\right)=f'(\xi )\epsilon _n+\frac{1}{2}f\text{''}(\xi )\epsilon _n^2-\left(f'(\xi )\epsilon
_{n-1}+\frac{1}{2}f\text{''}(\xi )\epsilon _{n-1}^2\right)=f'(\xi )\left(\epsilon _n-\epsilon _{n-1}\right)+\frac{1}{2}f\text{''}(\xi )\left(\epsilon
_n^2-\epsilon _{n-1}^2\right)\)\\
making our secant method equation\\
\(\epsilon _{n+1}=\epsilon _n-\left(f'(\xi )\epsilon _n+\frac{1}{2}f\text{''}(\xi )\epsilon _n^2\right)\frac{\epsilon _n-\epsilon _{n-1}}{f'(\xi
)\left(\epsilon _n-\epsilon _{n-1}\right)+\frac{1}{2}f\text{''}(\xi )\left(\epsilon _n^2-\epsilon _{n-1}^2\right)}\)\\
let{'}s look at the fraction\\
\(\frac{\epsilon _n-\epsilon _{n-1}}{f'(\xi )\left(\epsilon _n-\epsilon _{n-1}\right)+\frac{1}{2}f\text{''}(\xi )\left(\epsilon _n^2-\epsilon _{n-1}^2\right)}=\frac{1}{f'(\xi
)+\frac{1}{2}f\text{''}(\xi )\frac{\left(\epsilon _n^2-\epsilon _{n-1}^2\right)}{\epsilon _n-\epsilon _{n-1}}}=\frac{1}{f'(\xi )}\frac{1}{1+\frac{1}{2}\frac{f\text{''}(\xi
)}{f'(\xi )}\frac{\left(\epsilon _n^2-\epsilon _{n-1}^2\right)}{\epsilon _n-\epsilon _{n-1}}}\)\\
Since the second term in the denominator is so small, we can take the approximation \(\frac{1}{1+x}\to  1-x\) leading to\\
\(\frac{1}{f'(\xi )}\frac{1}{1+\frac{1}{2}\frac{f\text{''}(\xi )}{f'(\xi )}\frac{\left(\epsilon _n^2-\epsilon _{n-1}^2\right)}{\epsilon _n-\epsilon
_{n-1}}}=\frac{1}{f'(\xi )}\left(1-\frac{1}{2}\frac{f\text{''}(\xi )}{f'(\xi )}\frac{\left(\epsilon _n^2-\epsilon _{n-1}^2\right)}{\epsilon _n-\epsilon
_{n-1}}\right)\)\\
our recurrence relation becomes\\
\(\epsilon _{n+1}=\epsilon _n-\left(f'(\xi )\epsilon _n+\frac{1}{2}f\text{''}(\xi )\epsilon _n^2\right)\frac{1}{f'(\xi )}\left(1-\frac{1}{2}\frac{f\text{''}(\xi
)}{f'(\xi )}\frac{\left(\epsilon _n^2-\epsilon _{n-1}^2\right)}{\epsilon _n-\epsilon _{n-1}}\right)\\
\\
\epsilon _{n+1}=\epsilon _n-\left(\epsilon _n+\frac{1}{2}\frac{f\text{''}(\xi )}{f'(\xi )}\epsilon _n^2\right)\left(1-\frac{1}{2}\frac{f\text{''}(\xi
)}{f'(\xi )}\frac{\left(\epsilon _n^2-\epsilon _{n-1}^2\right)}{\epsilon _n-\epsilon _{n-1}}\right)\\
\\
\epsilon _{n+1}= \left(\frac{1}{2}\frac{f\text{''}(\xi )}{f'(\xi )}\right)^2 \epsilon _n^3+\left(\frac{1}{2}\frac{f\text{''}(\xi )}{f'(\xi )}\right)\epsilon
_n\epsilon _{n-1} +\left(\frac{1}{2}\frac{f\text{''}(\xi )}{f'(\xi )}\right)^2\epsilon _n^2\epsilon _{n-1}\)\\
We can neglect the first and third terms since \(\epsilon ^3\) will be very small and thus negligible, leaving\\
\(\epsilon _{n+1}\approx \left(\frac{1}{2}\frac{f\text{''}(\xi )}{f'(\xi )}\right)\epsilon _n\epsilon _{n-1}\)\\
we can now change the LHS and solve\\
\(C \epsilon _n^r=\left(\frac{1}{2}\frac{f\text{''}(\xi )}{f'(\xi )}\right)\epsilon _n\epsilon _{n-1}\to  \frac{ \epsilon _n^r}{\epsilon _n}=\frac{1}{C}\left(\frac{1}{2}\frac{f\text{''}(\xi
)}{f'(\xi )}\right)\epsilon _{n-1}\\
\\
 \epsilon _n^{r-1}=\frac{1}{C}\left(\frac{1}{2}\frac{f\text{''}(\xi )}{f'(\xi )}\right)\epsilon _{n-1}\to  \epsilon _n=\left(\frac{1}{2C}\frac{f\text{''}(\xi
)}{f'(\xi )}\right)^{\frac{1}{r-1}}\epsilon _{n-1}^{\frac{1}{r-1}}\)\\
Once again, we can plug in the LHS leading to\\
\(\text{C$\epsilon $}_{n-1}^r=\left(\frac{1}{2C}\frac{f\text{''}(\xi )}{f'(\xi )}\right)^{\frac{1}{r-1}}\epsilon _{n-1}^{\frac{1}{r-1}}\)\\
This implies then\\
\(C=\left(\frac{1}{2C}\frac{f\text{''}(\xi )}{f'(\xi )}\right)^{\frac{1}{r-1}}\\
\\
r=\frac{1}{r-1}\)\\
We know the order of convergence must be a positive number, so solving for r yields,\\
\(r(r-1)=1\\
\\
r^2-r-1=0\\
\\
r=\frac{-(-1)+\sqrt{1-4(1)(-1)}}{2}=\frac{1+\sqrt{5}}{2}=1.61803\)\\
so the order of convergence is the Golden Ratio, \(r=\phi =1.618\)

\end{document}
